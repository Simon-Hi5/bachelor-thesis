% Einleitung

Das Ziel dieser Bachelorarbeit ist es mit neuen Methoden und Werkzeugen eine moderne Webanwendung zu gestalten. Dabei soll DevOps-Bezogen der gesamte Softwareprozess von der Entwicklung bis zum Betrieb analysiert und umgesetzt werden.

\section{Theoretischer Rahmen}

DevOps, Cloud und Container Microservices und Kubernetes

Kubernetes ist der Branchenstandard und die Grundlage für moderne Webanwendungen (\cite[Vorwort]{arundelCloud2019}).

Die Verwendungsweise wird maßgeblich beeinflusst durch neue Technologien. (\cite[S. 16]{newmanMicroservices2015})

DevOps, Kubernetes und Microservices sind zwar komplett unterschiedliche Dinge haben jedoch zweifelsfrei Verbindungen und können in Zusammenarbeit ihre Stärken bestmöglich ausnutzen.

\subsection{DevOps}

Das Kofferwort DevOps ist eine Sammlung von Lösungsansätzen wie IT-Leistungen schnell und risikoarm bereitgestellt werden können. Wie das Kofferwort "DevOps" bereits andeutet, bewirkt es eine stärkere Verschmelzung zwischen Softwareentwicklung (Development) und IT-Betrieb (Operations). Dabei ist DevOps nich klar definiert. DevOps stellt den Kundennutzen in den Mittelpunkt und erinnert dabei an agile Softwareentwicklungsmethoden. DevOps und aglie Entwicklung schließen sich auch nicht gegenseitig aus unterscheiden sich jedoch durch die einbezogenen Gruppen und Abteilungen. Den während DevOps bezieht sich wie bereits erklärt nicht nur auf den Entwicklungsprozess und die Softwareentwickler sondern auf den gesamten Softwareprozess.

Um DevOps in Unternehmen umzusetzen reicht es nicht die entsprechenden Werkzeuge einzuführen, sondern es muss zu einem Kulturwandel kommen.

DevOps ist eine Sammlung an Methoden und Denkweisen, welche sich immer weiter verbreiten. Das Vorgehen in der Fallstudie dieser Arbeit kommen auch DevOps-Werkzeuge zum Einsatz. Da die gewählten Technologien und Architekturmuster sich nicht nur auf den Entwurf sondern auch


Eine der wichtigsten Werkzeuge von DevOps ist Continuous Delivery.

\subsection{Microservices}

Im Mittelpunkt dieser Arbeit stehen Microservices. Microservices sind

 Bei Microservices handelt es sich um ein Architekturmuster zur Modualisierung von Software (\cite[S. 15]{newmanMicroservices2015}). Obwohl der Begriff Microservices noch realtiv jung ist, sind die dahinterstehenden Konzepte bereits älter (\cite[S. 15]{newmanMicroservices2015}). Große Systeme werden schon lange in kleine Module unterteilt. Die Besonderheit von Microservices liegt darin, dass die Module einzelne Programme sind. Ein solches einzelnes Programme wird als Microservice bezeichnet. Der Begriff Microservice ist nicht fest definiert (\cite[S. 2]{wolffMicroservices2018}), deshalb werden im nachfolgenden Kapitel die wichtigsten Eigenschaften und Merkmale betrachtet.


Microservices sind ein Architekturmuster. Architekturmuster beschreiben die Grundstruktur von Systemen in der Softwareentwicklung. Microserices werden als Architekturmuster in die Kategorie der verteilten Systeme eingeordnet. Die einzelnen Microservices laufen zumeist auf vielen unterschiedlichen Rechnern. Die Microservices sind dabei voneinander unabhängig und kommunizieren in einem Netzwerk über festgelegte Schnittstellen miteinander.
 
Architekturmuster Erklärung
Wenn das Architekturmuster für ein System bestimmend ist, nennt man es auch dessen Architekturstil

Erklärung Monolith


 Microservices sind ein Architekturmuster, welches konträr zu einer klassischen monolithischen Architektur ist. Dabei heben sich Microservices durch ihre lose Kopplung und die damit einhergehende erhöhte Unabhängigkeit hervor. Obwohl Microservices ein neuer Begriff sind, ist das Konzept schon deutlich länger vorhanden.

Microservices vs Monolithen

Dabei werden Microservices häufig im geschäftlichen Umfeld eingesetzt (\cite[S. 15]{newmanMicroservices2015}). 

Marktwachstumsprognose: https://www.instanttechnews.com/technology-news/2020/02/16/cloud-microservices-market-2020-trends-market-share-industry-size-opportunities-analysis-and-forecast-by-2026/

Microservices machen Gebrauch von den neu entwickelten Technologien und Verfahren des letzten Jahrzehnts. Vor allem mit DevOps-Methoden und Kubernetes können ...

\subsubsection{Merkmale}

Microservices sind eigenständige Programme welche über ein Neztwerk miteinander kommunizieren.

\paragraph{Größe}

Microservices sollen nur eine Aufgabe erledigen, diese jedoch bestmöglich. Dieser Ansatz ist nicht neu und entstammt der UNIX-Philosophie: "Mache nur eine Sache und mache sie gut" (Douglas McIlroy).
Wie der Name "Microservices" bereits andeutet, handelt es sich dabei offensichtlich um kleine Services. Eine genaue Festlegung wie groß die Services sein sollten gibt es jedoch nicht. Die Anzahl der Codezeilen (Lines of Code) können einen Hinweis geben, jedoch sind derartige Kriterien stark von der Programmiersprache und dem verwendeten Technologie-Stack abhängig. Stattdessen sollte sich die Größe an fachliche Gegebenheiten anpassen. Je kleiner die Services gestaltet werden, umso stärker kommen die in den nachfolgenden Abschnitten beschriebenen Vor- und Nachteile zur Geltung. Des Weiteren sollte ein Microservice nur so Groß sein, dass er von einem einzigen Entwicklerteam betreut werden kann.

Der Name "Microservices" deutet schon an, dass es sich offensichtlich um kleine Services handelt.
Eine objektive Messung der Größe ist jedoch nicht möglich. Um die Größe eines Microservices zu bestimmen, könnte man beispielsweise die Anzahl der Codezeilen (LoC) verwenden. Doch die Lines of Code hängen starkt von der verwendeten Programmiersprache und dem Technologie-Stack ab. Eine Messung der Größe nach rein objektiven Kriterien macht demnach keinen Sinn (\cite[S. 31]{wolffMicroservices2018})(\cite[S. 22]{newmanMicroservices2015}). Stattdessen sollte sich die Größe an die fachlichen Gegebenheiten richten. Je kleiner die Services werden desto mehr kommen die in den nächsten Abschnitten beschriebenen Vor- und Nachteile zur Geltung. Ein Microservice sollte von einem einzigen Enticklerteam gehandhabt werden (\cite[S. 23]{newmanMicroservices2015}). Falls das nicht mehr möglich ist, könnte es darauf hindeuten, dass der Microservice zu groß ist.

\paragraph{Sepzialisierung}



\paragraph{Eigenständigkeit}

Microservices laufen als eigenständige Programme und müssen unabhängig voneinander deploybar sein. Jeder Microservice ist ein eigener Prozess, welcher isoliert von den anderen abläuft. Die Isolierung trägt dazu bei, dass verteilte System besser zu verstehen aber auch keine unbemerkten Abhängigkeiten zwischen den Services entstehen zu lassen. Neue Technologien wie Containervirtualisierung können dies erleichtern. Die Kommunikation der Services erfolgt über ein Netzwerk und mittels sprachunabhängiger Schnittstellen (API).

Eine ausreichende Eigenständigkeit ist nur gegeben, sobald die Services unabhängig voneinander verändert und bereitgestellt werden können. Ein Entwicklerteam sollte sich bei der Implementierung von Änderungen oder neuen Funktionen nicht mit anderen Teams absprechen müssen. 

\paragraph{Ersetzbarkeit}



\subsubsection{Vorteile}

Das Aufteilen von Software bringt wichtige Vorteile mit sich.

\paragraph{Modularisierung}
Bei klassischen Software-Monolithen, welche aus Komponenten zusammengestellt wird, entstehen schnell unerwünschte Abhängigkeiten. Die viele Abhängigkeiten erschweren die Wartung oder Weiterentwicklung. Da die einzelnen Microservices eigene Programme sind, herrscht eine starke Modularisierung. Die Programme sind eigenständig und kommunizieren nur über explizite Schnittstellen. Ungewollte Abhängigkeiten entstehen hier deutlich schwerer. (\cite[S. 3]{wolffMicroservices2018})

In der Praxis wird die Architektur von Deployment-Monolithen meistens zunehmend schlechter. (\cite[S. 3]{wolffMicroservices2018})

\paragraph{Ersetzbarkeit}
Da Microservices nur über eine explizite Schnittstelle genutzt werden, können sie einfach durch einen Service, der die selbe Schnittstelle anbietet ersetzt werden. Bei der Ersetzung ist der neue Service nicht an den Technologie-Stack des alten Service gebunden. Auch die Risiken werden geringer, da bei schwerwiegenden Fehlentscheidungen in der Entwicklung, ein Austausch mit weniger Aufwand verbunden ist. (\cite[S. 4]{wolffMicroservices2018})

Sie können auch deutlich schneller eingesetzt werden. (Time to Market)

\paragraph{Skalierbarkeit}
Durch die Unabhängigkeit der Microservices können sie auch unabhängig voneinander skaliert werden. So kann eine einzelne Funktionalität, welche stärker genutzt wird, skaliert werden, ohne das gesamte System zu skalieren. (\cite[S. 5]{wolffMicroservices2018})

\paragraph{Technologiefreiheit}

Des Weiteren führt die Unabhängigkeit auch zu einer großen Technologiefreiheit. Die verwendeten Technologien müssen schließlich nur in der Lage sein die explizite Schnittstelle anzubieten. (\cite[S. 5]{wolffMicroservices2018})



\paragraph{Continuous Delivery}

Ein wesentlicher Grund für die Einführung ist Continuous Delivery. Die kleinen Microservices können leichter deployt werden und das Deployment bietet weniger Gefahren und ist einfacher abzusichern, als bei einem Monolithen. (\cite[S. 5]{wolffMicroservices2018})

\subsubsection{Herausforderungen}

\paragraph{Versteckte Beziehungen}

\paragraph{Refactoring}

\paragraph{Fachliche Architektur}

\paragraph{Komplexität}

\paragraph{Verteilte Systeme}

\subsubsection{Architektur}

\subsubsection{Integration}

\subsubsection{Bereitstellung / Deployment und Betrieb}

\subsubsection{Technologien}

\subsection{Containervirtualisierung}

\subsection{Kubernetes}
