% Einleitung

Das Ziel dieser Bachelorarbeit ist es mit neuen Methoden und Werkzeugen eine moderne Webanwendung zu gestalten. Dabei soll DevOps-Bezogen der gesamte Softwareprozess von der Entwicklung bis zum Betrieb analysiert und umgesetzt werden.

\section{Theoretischer Rahmen}

DevOps, Cloud und Container Microservices und Kubernetes

Kubernetes ist der Branchenstandard und die Grundlage für moderne Webanwendungen (\cite[Vorwort]{arundelCloud2019}).

Die Verwendungsweise wird maßgeblich beeinflusst durch neue Technologien. (\cite[S. 16]{newmanMicroservices2015})

DevOps, Kubernetes und Microservices sind zwar komplett unterschiedliche Dinge haben jedoch zweifelsfrei Verbindungen und können in Zusammenarbeit ihre Stärken bestmöglich ausnutzen.

\subsection{DevOps}

Das Kofferwort DevOps ist eine Sammlung von Lösungsansätzen wie IT-Leistungen schnell und risikoarm bereitgestellt werden können. Wie das Kofferwort "DevOps" bereits andeutet, bewirkt es eine stärkere Verschmelzung zwischen Softwareentwicklung (Development) und IT-Betrieb (Operations). Dabei ist DevOps nich klar definiert. DevOps stellt den Kundennutzen in den Mittelpunkt und erinnert dabei an agile Softwareentwicklungsmethoden. DevOps und aglie Entwicklung schließen sich auch nicht gegenseitig aus unterscheiden sich jedoch durch die einbezogenen Gruppen und Abteilungen. Den während DevOps bezieht sich wie bereits erklärt nicht nur auf den Entwicklungsprozess und die Softwareentwickler sondern auf den gesamten Softwareprozess.

Um DevOps in Unternehmen umzusetzen reicht es nicht die entsprechenden Werkzeuge einzuführen, sondern es muss zu einem Kulturwandel kommen.

DevOps ist eine Sammlung an Methoden und Denkweisen, welche sich immer weiter verbreiten. Das Vorgehen in der Fallstudie dieser Arbeit kommen auch DevOps-Werkzeuge zum Einsatz. Da die gewählten Technologien und Architekturmuster sich nicht nur auf den Entwurf sondern auch


Eine der wichtigsten Werkzeuge von DevOps ist Continuous Delivery.

\subsection{Microservices}

Im Mittelpunkt dieser Arbeit stehen Microservices. Microservices sind ein Architekturmuster der konträr zu einer klassischen monolithischen Architektur ist. Dabei heben sich Microservices durch ihre lose Kopplung und die damit einhergehende erhöhte Unabhängigkeit hervor. Obwohl Microservices ein neuer Begriff sind, ist das Konzept schon deutlich länger vorhanden.

Microservices vs Monolithen

Dabei werden Microservices häufig im geschäftlichen Umfeld eingesetzt (\cite[S. 15]{newmanMicroservices2015}). 

Marktwachstumsprognose: https://www.instanttechnews.com/technology-news/2020/02/16/cloud-microservices-market-2020-trends-market-share-industry-size-opportunities-analysis-and-forecast-by-2026/

Microservices machen Gebrauch von den neu entwickelten Technologien und Verfahren des letzten Jahrzehnts. Vor allem mit DevOps-Methoden und Kubernetes können ...

\subsubsection{Merkmale}



\subsubsection{Vorteile}

Das Aufteilen von Software bringt wichtige Vorteile mit sich.

\paragraph{Modularisierung}
Bei klassischen Software-Monolithen, welche aus Komponenten zusammengestellt wird, entstehen schnell unerwünschte Abhängigkeiten. Die viele Abhängigkeiten erschweren die Wartung oder Weiterentwicklung. Da die einzelnen Microservices eigene Programme sind, herrscht eine starke Modularisierung. Die Programme sind eigenständig und kommunizieren nur über explizite Schnittstellen. Ungewollte Abhängigkeiten entstehen hier deutlich schwerer. (\cite[S. 3]{wolffMicroservices2018})

In der Praxis wird die Architektur von Deployment-Monolithen meistens zunehmend schlechter. (\cite[S. 3]{wolffMicroservices2018})

\paragraph{Ersetzbarkeit}
Da Microservices nur über eine explizite Schnittstelle genutzt werden, können sie einfach durch einen Service, der die selbe Schnittstelle anbietet ersetzt werden. Bei der Ersetzung ist der neue Service nicht an den Technologie-Stack des alten Service gebunden. Auch die Risiken werden geringer, da bei schwerwiegenden Fehlentscheidungen in der Entwicklung, ein Austausch mit weniger Aufwand verbunden ist. (\cite[S. 4]{wolffMicroservices2018})

Sie können auch deutlich schneller eingesetzt werden. (Time to Market)

\paragraph{Skalierbarkeit}
Durch die Unabhängigkeit der Microservices können sie auch unabhängig voneinander skaliert werden. So kann eine einzelne Funktionalität, welche stärker genutzt wird, skaliert werden, ohne das gesamte System zu skalieren. (\cite[S. 5]{wolffMicroservices2018})

\paragraph{Technologiefreiheit}

Des Weiteren führt die Unabhängigkeit auch zu einer großen Technologiefreiheit. Die verwendeten Technologien müssen schließlich nur in der Lage sein die explizite Schnittstelle anzubieten. (\cite[S. 5]{wolffMicroservices2018})

\paragraph{Continuous Delivery}

Ein wesentlicher Grund für die Einführung ist Continuous Delivery. Die kleinen Microservices können leichter deployt werden und das Deployment bietet weniger Gefahren und ist einfacher abzusichern, als bei einem Monolithen. (\cite[S. 5]{wolffMicroservices2018})

\subsubsection{Herausforderungen}

\paragraph{Versteckte Beziehungen}

\paragraph{Refactoring}

\paragraph{Fachliche Architektur}

\paragraph{Komplexität}

\paragraph{Verteilte Systeme}

\subsubsection{Architektur}

\subsubsection{Integration}

\subsubsection{Bereitstellung / Deployment und Betrieb}

\subsubsection{Technologien}

\subsection{Containervirtualisierung}

\subsection{Kubernetes}
