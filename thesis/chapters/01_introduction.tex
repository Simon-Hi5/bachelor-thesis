\section{Einleitung}

Kurzbeschreibung \\
Gliederung \& Struktur der Arbeit


\subsection{Motivation}
Durch Container, Microservices und Kubernetes hat sich in den letzten Jahren ein erheblicher
Wandel in der IT-Branche vollzogen. Containervirtualisierung erleichtert das Bauen großer
verteilter Systeme durch das Verbinden von vielen kleinen Services. Im Gegensatz zu
monolithischen Anwendungen erleichtern Microservices den Einsatz von verschiedenen
Technologien, die Skalierung und den modularen Aufbau (Newman et al., 2015, S. 24 ff.).
Um die große Anzahl an Services zu steuern, kommen Orchestrierungssysteme zum Einsatz.
In diesem Bereich hat sich in den letzten Jahren Kubernetes zum Standard entwickelt. Das
alles sind Trends die DevOps unterstützen und für ein zunehmendes Verschmelzen von
Softwareentwicklern und Systemadministratoren sorgen. Heutzutage sind DevOps-Prinzipien
tief in modernen Anwendungen verwurzelt und haben erhebliche Auswirkungen auf alle
Phasen des Entwicklungszyklus. Vor allem der Entwurf und die Bereitstellung wurden enorm
beeinflusst und haben sich grundsätzlich verändert. Den containerisierten, verteilten Systemen
gehört die Zukunft (Arundel und Domingus, 2019, S. 1). In dieser Bachelorarbeit wird deshalb
der Fokus auf dem Entwurf und der Bereitstellung von modernen Microservice-basierten
Anwendungen mit dem Branchenstandard Kubernetes liegen.


\subsection{Zielsetzung}
Die Bachelorarbeit widmet sich dem Entwurf und der Bereitstellung von Microservices mit
Kubernetes. Die Arbeit wird zuerst die nötigen Methoden und Technologien beschreiben, um
diese im Anschluss anhand einer konkreten Fallstudie einzusetzen. Das Ziel der Fallstudie
ist es, ein auf Microservices basierendes Customer-Relationship-Management-System (CRM-
System) zu entwerfen und dieses mithilfe von Kubernetes bereitzustellen. Dabei soll ein
Verfahren, das dem aktuellen Stand der Technik entspricht, vom Entwurf bis zur Bereitstellung
von modernen Microservice-basierten Anwendungen implementiert werden.

\subsection{Abgrenzung}
Die folgenden Aspekte werden nicht betrachtet:
\begin{itemize}
\item Sicherheit (Authentifizierung)
\item Cloud Computing
\end{itemize}
