\section{Einleitung}

\dictum[{ Louis Pasteur \parencite[S. 348]{pasteurOeuvres1933}}]{\center ``Veränderungen begünstigen nur den, der darauf vorbereitet ist." \upshape{} }
\vspace{1em} 

Die IT-Branche befindet sich in einem erheblichen Wandel. Eine Reihe von neuen Methoden und Werkzeugen revolutionieren die Software-Welt. Angefangen hat es mit der Verbreitung von Cloud Computing und den damit einhergehenden großen verteilten Systemen. Diese machten es immer schwieriger den Betrieb von der Architektur eines Systems zu trennen. Der DevOps-Ansatz soll dieses Problem durch eine bessere Zusammenarbeit von Softwareentwicklung und IT-Betrieb beheben. Eng verwoben mit den Grundsätzen von DevOps ist das neue Architekturmuster der Microservices, bei dem große Anwendungen in kleine unabhängige Services aufgeteilt werden. Werkzeuge zur Containervirtualisierung wie Docker helfen dabei, die einzelnen Services auf verschiedenen Rechnern bereitzustellen und Kubernetes unterstütze bei der Steuerung der riesigen Containermengen. Zusammen bilden all diese Innovationen den Grundstein für moderne Anwendungen, containerisierten Microservices, welche mit Kubernetes verwaltet werden. In dieser Arbeit wird die Kombination dieser Innovationen genauer betrachtet, um die Möglichkeiten und Herausforderungen vom Entwurf bis zur Bereitstellung kennenzulernen. Zu Beginn wird in diesem Kapitel die Motivation, die Zielsetzung sowie der Aufbau der Arbeit beschrieben.

\subsection{Motivation}

Container haben sich etabliert und sind in der heutigen IT-Landschaft nicht mehr wegzudenken. Google stellt fast alle seine Dienste in Containern bereit und startet so über zwei Milliarden Container pro Woche \parencite[vgl.][S. 43]{liebelSkalierbare2021}. Kubernetes ist mittlerweile der Branchenstandard zur Containerorchestrierung und die Grundlage für moderne Webanwendungen \parencite[vgl.][Vorwort]{arundelCloud2019}. In IT-Projekten von Unternehmen, in denen Software zur Containerorchestrierung eingesetzt wird, setzen 91\% auf Kubernetes \parencite[vgl.][S. 8]{cloudnativecomputingfoundationCloud2020}. Grundlegende Fähigkeiten in diesen Bereichen sind heutzutage unabdingbar.

Der Übergang zu Microservice-Architekturen ist in vielen Unternehmen in vollem Gange. Immer mehr monolithische Anwendungen werden so in Microservices aufgespalten. Das Marktvolumen für Microservices in der Cloud wurde 2020 auf 831 Millionen \ac{USD} geschätzt. Bis zum Jahre 2026 soll der Markt mit einer durchschnittlichen jährlichen Wachstumsrate von 21.7\%  auf 2701 Millionen \ac{USD} anwachsen \parencite[vgl.][S. 7]{mordorintelligenceGlobal2020}.

Während sich Docker und Kubernetes schon feste Größen sind, befinden sich Microservices gerade in einem großen Trend. Ein abflachen dieses Trends ist nicht zu erkennen. Die beiden Softwareentwickler Brendan Burns und David Oppenheimer, welche Kubernetes mitentwickelten, halten das Konzept von containersierten Microservices sogar für ähnlich revolutionär, wie die Popularisierung der objektorientierten Programmierung \parencite[vgl.][S. 1]{burnsDesign2016}. Der Cloud-Experte John Arundel denkt, dass aufgrund dieser Revolutionen die Zukunft in containersierten verteilten Systemen liegt, die auf der Kubernetes-Plattform laufen \parencite[vgl.][S. 1]{arundelCloud2019}. Fähgikeiten in diesen Bereichen sind somit sehr gefragt und werden Unternehmen gut entlohnt. Die Kombination dieser Methoden und Werkzeugen ergänzt sich perfekt und ist die Zukunft für große Systeme. Jedoch sind die Technologien diffizil und bringen neben zahlreichen Vorteilen auch viele Herausforderungen mit sich. Es ist von großer Bedeutung die Technologien in ihrer Gesamtheit zu verstehen und anwenden zu können. Diese Arbeit wird sich deshalb dem Entwurf und der Bereitstellung von Microservices mit Kubernetes widmen.
  
\subsection{Zielsetzung}
Das Ziel dieser Bachelorarbeit ist es, eine moderne Webanwendung, nach aktuellem Stand der Technik, mit einer Microservice-Architektur zu entwerfen und mithilfe von Kubernetes bereitzustellen. Dazu soll zuerst der theoretische Rahmen erläutert werden, um anschließend eine Fallstudie durchzuführen. Die Fallstudie wird am Beispiel eines \ac{CRM}-Systems durchgeführt. In der Fallstudie soll ein Verfahren vom Entwurf bis zur Bereitstellung implementiert werden. Um Aussagen zum Anwendungsgebiet und der Implementierung von containerisierten Microservices mit Kubernetes zu treffen, sollen geklärt werden, 

\begin{itemize}
\item welche Vorteile und Herausforderungen Microservices bieten,
\item wobei Containervirtualisierung sowie Kubernetes den Einsatz von Microservices unterstützt,
\item wie eine Microservice-Architektur entworfen werden kann,
\item wie Microservices mit Kubernetes bereitgestellt werden können.
\end{itemize}

\subsection{Aufbau der Arbeit}

Als Erstes wird in Kapitel 2 der theoretische Rahmen der Arbeit erläutert. Es wird DevOps, das Architekturmuster der Microservices, Docker sowie Kubernetes genauer erklärt. Auf Basis dieser theoretischen Grundlagen wird die Fallstudie durchgeführt. In Kapitel 3 wird zuerst die Problemstellung beschrieben. Danach wird in Kapitel 4 der Entwurf und in Kapitel 5 die Implementierung der Microservices dargelegt. Anschließend wird in Kapitel 6 die Bereitstellung mit Kubernetes erklärt. Zum Schluss wird in Kapitel 7 ein Fazit gezogen und die Ergebnisse diskutiert.


