\begin{abstract}

\section*{Zusammenfassung}
Beim Entwurf von modernen Anwendungen etabliert sich das Architekturmuster der Microservices. Microservices teilen große Anwendungen in kleine unabhängige Services auf. Kubernetes, eine Software zur Containerorchestrierung, hilft dabei die vielen Microservice bereitzustellen.

 In der vorliegenden Bachelorarbeit wird der Entwurf und die Bereitstellung von Microservices mit Kubernetes analysiert, um Aussagen zur Umsetzung und dem Anwendungsgebiet treffen zu können.

Im Ersten Teil der Arbeit werden dazu die theoretischen Grundlagen bezüglich Microservices, Containervirtualisierung und Kubernetes dargelegt. Darauf aufbauend erfolgt die Umsetzung eines Fallbeispiels. Im Fallbeispiel wird ein Verfahren vom Entwurf bis zur Bereitstellung containerisierter Microservices mit Kubernetes implementiert. Durch den theoretischen Hintergrund und die praktische Anwendung können die Vorteile und Herausforderungen besser eingeschätzt werden.



\end{abstract}

\begin{abstract}

\section*{Abstract}


\end{abstract}